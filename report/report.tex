\documentclass[11pt]{article}

\usepackage{amsmath}
\usepackage{amssymb}
\usepackage{graphicx}
\usepackage{booktabs}
\usepackage{hyperref}
\usepackage{geometry}
\geometry{margin=1in}

\title{Sentiment Classification with Word Embeddings:\\
A Study on Negation-Aware Representations}
\author{Chen Zile}
\date{\today}

\begin{document}
\maketitle

\section{Introduction}

Sentiment classification is a core task in natural language processing, aiming to determine whether a piece of text expresses positive or negative sentiment.
In this project, we focus on binary sentiment classification of Twitter data using word embeddings and classical machine learning classifiers.

A common approach represents each sentence by aggregating word embeddings (e.g., by averaging) and then applies a linear classifier.
Although simple and efficient, this approach ignores important linguistic phenomena such as negation.
For instance, phrases like \textit{``not good''} or \textit{``never liked''} convey negative sentiment despite containing positive words.

In this work, we systematically evaluate several embedding-based classifiers with different aggregation strategies.
We observe that more advanced weighting schemes and classifiers improve performance over a simple baseline, but the best results are obtained when negation information is explicitly incorporated into the sentence representation.

\section{Dataset and Preprocessing}

We use the Twitter sentiment dataset provided in the course project.
The dataset consists of short tweets labeled as either positive or negative.
All tweets are pre-tokenized and lowercased.

The dataset is split into training and validation sets, with approximately 90\% of the data used for training and 10\% for validation.
A vocabulary is constructed from the training data and stored in a serialized file.
Words outside the vocabulary are ignored during feature construction.

Word embeddings are trained using a GloVe-style model on the training corpus.
Unless otherwise specified, we use embeddings of dimension 50.
The learned embeddings are fixed and shared across all classifiers to ensure a fair comparison.

\section{Methods}

\subsection{Baseline Representation}

In the baseline approach, each tweet is represented by the average of the embeddings of the words it contains.
Formally, for a tweet with words $\{w_1, \dots, w_n\}$ and embeddings $\{\mathbf{v}_1, \dots, \mathbf{v}_n\}$, the sentence representation is

\[
\mathbf{s} = \frac{1}{n} \sum_{i=1}^{n} \mathbf{v}_i.
\]

This representation ignores word order and treats all words equally.

\subsection{Embedding-Based Classifiers}

Using the same sentence representation, we evaluate three classifiers:

\begin{itemize}
    \item \textbf{Logistic Regression}: A linear classifier trained on averaged embeddings.
    \item \textbf{TF-IDF Weighted Embeddings}: Word embeddings are weighted by their TF-IDF scores before aggregation.
    \item \textbf{Linear SVM}: A linear support vector machine trained on averaged embeddings.
\end{itemize}

These methods differ in how they weight or separate features, but none of them explicitly models negation.

\subsection{Negation-Aware Representation}

Negation plays a crucial role in sentiment analysis, as it can reverse the polarity of sentiment-bearing words.
To model this phenomenon, we introduce a negation-aware sentence representation.

When a negation word (e.g., \textit{``not''}, \textit{``no''}, \textit{``never''}) is encountered, the embeddings of the following words within a fixed window are modified by reversing their sign.
If a word $w_i$ falls within the scope of a negation, its embedding $\mathbf{v}_i$ is replaced by $-\mathbf{v}_i$.

The final tweet representation is obtained by averaging these modified embeddings.
This approach preserves the embedding space while explicitly encoding sentiment reversal.

\section{Experimental Results}

We evaluate all methods using classification accuracy on both the training and validation sets.
The results with embedding dimension 50 are summarized in Table~\ref{tab:results}.

\begin{table}[h]
\centering
\begin{tabular}{lcc}
\toprule
\textbf{Method} & \textbf{Train Accuracy} & \textbf{Validation Accuracy} \\
\midrule
Baseline (Logistic Regression) & 0.6665 & 0.6672 \\
TF-IDF Weighted Embeddings & 0.7181 & 0.7225 \\
Linear SVM & 0.7390 & 0.7406 \\
Negation-Aware Classifier & \textbf{0.7429} & \textbf{0.7421} \\
\bottomrule
\end{tabular}
\caption{Classification accuracy of different methods (embedding dimension = 50)}
\label{tab:results}
\end{table}

The baseline model achieves the lowest performance.
TF-IDF weighting and SVM classifiers significantly improve accuracy, indicating that better feature weighting and margin-based classification are beneficial.
The negation-aware representation consistently achieves the best performance on both training and validation sets.

\section{Discussion}

The experimental results suggest that sentence representation plays a more important role than the choice of classifier alone.
While TF-IDF weighting and SVMs provide notable improvements over simple averaging, explicitly modeling negation yields the highest accuracy.

We further observe that embedding dimensionality has little impact on the baseline mean-pooling method, but substantially affects TF-IDF weighted and negation-aware representations.
This indicates that simple averaging compresses information aggressively and fails to exploit additional representational capacity offered by higher-dimensional embeddings.
In contrast, methods that selectively weight or modify embeddings can better utilize the structure of high-dimensional embedding spaces.

Overall, the negation-aware representation offers a strong performance�Ccomplexity trade-off, achieving the best results with minimal additional modeling assumptions.

\section{Conclusion}

In this project, we studied sentiment classification using word embeddings and classical machine learning models.
We showed that simple baseline methods based on mean pooling are limited in performance.

By incorporating TF-IDF weighting, stronger classifiers, and explicit negation handling, we achieved progressively better results.
Among all methods, the negation-aware representation consistently performed best, demonstrating the importance of incorporating linguistic insights into sentence representations.

Future work could explore more precise modeling of negation scope or combine negation handling with contextual or syntactic information.

\end{document}